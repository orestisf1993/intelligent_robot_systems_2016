\section{Extra Challenge 2: Βελτίωση υπολογισμών στο target selection}
Στα πλαίσια αυτού του στόχου πραγματοποιήθηκαν διάφορες αλλαγές που αποσκοπούν στη βελτίωση της απόδοσης των υπολογισμών που σχετίζονται με τις λειτουργίες του αρχείου
\path{./art_autonomous_exploration/src/target_selection.py}.
Το σύνολο των αλλαγών που πραγματοποιήθηκαν μπορούν να βρεθούν με την εντολή:\\
%\mintinline{bash}!git diff -w acc7a81 "8e1cf67^1" -- art_autonomous_exploration/src/*.py!%TODO: check
\mintinline{bash}!git diff acc7a81 aa93f0c^1 -- art_autonomous_exploration/src/*.py!

Το πρώτο πρόβλημα που αντιμετωπίστηκε είναι αυτό της τελικής αντιγραφής των
αντικειμένων
\href{https://github.com/etsardou/intelligent_robot_systems_2016/blob/bc7bfee96e5598edc83a6bcadaaef982c2a398aa/art_autonomous_exploration/src/utilities.py#L44}{\mintinline{python}!brush!},
\href{https://github.com/etsardou/intelligent_robot_systems_2016/blob/bc7bfee96e5598edc83a6bcadaaef982c2a398aa/art_autonomous_exploration/src/utilities.py#L67}{\mintinline{python}!skeleton!} και
\href{https://github.com/etsardou/intelligent_robot_systems_2016/blob/bc7bfee96e5598edc83a6bcadaaef982c2a398aa/art_autonomous_exploration/src/utilities.py#L91}{\mintinline{python}!skeleton!}
στις συναρτήσεις
\mintinline{python}!brushfireFromObstacles()!,
\mintinline{python}!thinning()! και
\mintinline{python}!prune()!
της κλάσης
\mintinline{python}!Cffi!
στο αρχείο
\path{./art_autonomous_exploration/src/utilities.py}.
Τα αντικείμενα \mintinline{python}!ndarray! της \mintinline{python}!numpy! βιβλιοθήκης αποθηκεύονται σε ``C order'' ή ``row-major order''.
Δηλαδή, ένας δισδιάστατος πίνακας όπως αυτοί παραπάνω αποθηκεύεται σαν μονοδιάστατος, σε συνεχόμενες θέσης μνήμης.

\sloppy Ωστόσο, στο αρχείο
\path{./art_autonomous_exploration/src/cpp_functions.py}
που ορίζει τις συναρτήσεις γραμμένες σε κώδικα C χρησιμοποιούνται δισδιάστατοι πίνακες σε μορφή \mintinline{C}!int ** out!.
Για αυτό το λόγο η μεταφορά του numpy array σε δισδιάστατο πίνακα γίνεται ως εξής:
\begin{code}
\caption{Μεταφορά numpy array σε ffi}
\begin{pythoncode}
x = [np.array(v, dtype='int32') for v in array]
xi = ffi.new("int* [%d]" % (len(x)))
for i in range(len(x)):
    xi[i] = ffi.cast("int *", x[i].ctypes.data)
n = len(x)
m = len(x[0])
\end{pythoncode}
\end{code}
Για τον πίνακα \mintinline{C}!int ** input! των συναρτήσεων αυτό δε μας πειράζει αφού θέλουμε ούτως ή άλλως να αντιγράψουμε τα δεδομένα του \mintinline{python}!array!.
Ωστόσο, θα μπορούσαμε να αποφύγουμε την αντιγραφή του πίνακα \mintinline{C}!int ** output! στο \mintinline{python}!array!.

\sloppy Μια καλή προσέγγιση θα ήταν να ξαναγραφεί ο κώδικας των
\mintinline{C}!brushfireFromObstacles()!,
\mintinline{C}!thinning()! και
\mintinline{C}!prune()!
ώστε να δέχεται μονοδιάστατους πίνακες ως ορίσματα και να πραγματοποιεί την προσπέλασή τους με row major order.
Ένας πιο απλός τρόπος που δεν απαιτεί την αλλαγή ολόκληρου του κώδικα είναι η δημιουργία μιας συνάρτησης που παίρνει σαν όρισμα έναν μονοδιάστατο πίνακα κατευθείαν μέσω \mintinline{python}!array.ctypes.data!
και τον μετατρέπει σε δισδιάστατο (διπλό pointer) πίνακα χωρίς να πραγματοποιεί αντιγραφές.
\begin{code}
\caption{Μετατροπή μονοδιάστατου σε δισδιάστατου}
\begin{Ccode}
static int ** continuous_to_2D(int * old, int n, int m){
    int ** out_2d = malloc(n * sizeof(int*));
    int i;

    for (i = 0; i < n; i++){
        out_2d[i] = &old[i * m];
    }
    return out_2d;
}
\end{Ccode}
\end{code}
Με αυτό το τρόπο το \mintinline{C}!out_2d[i][j]! έχει την ίδια διεύθυνση στη μνήμη με το \mintinline{C}!old[i + m*j]! και το \mintinline{python}!array[i][j]! του αρχικού πίνακα numpy.

Στη συνέχεια, βελτιώθηκαν μερικοί υπολογισμοί πινάκων με χρήση της numpy~\cite{numpy}.
Στο \path{./art_autonomous_exploration/src/topology.py} οι αλλαγές για τον υπολογισμό της μεταβλητής \mintinline{python}!local! φαίνονται στη~\ref{lst:local}.
Επίσης, στη καταχώρηση~\ref{lst:viz},
η μεταβλητή \mintinline{python}!viz! και η κλήση της \mintinline{python}!RvizHandler.printMarker! μεταφέρθηκαν στη συνάρτηση \mintinline{python}!print_viz()! %TODO: name?
και βελτιώθηκε ο χρόνος υπολογισμού.

\begin{code}
\caption{Γρηγορότερος υπολογισμός skeleton πριν το thinning και pruning}\label{lst:local}
\begin{pythoncode}
-    local = numpy.zeros(ogm.shape)
-    for i in range(0, width):
-        for j in range(0, height):
-            if ogm[i][j] < 49:
-                local[i][j] = 1
+    local = (ogm < 49).astype("int32")
\end{pythoncode}
\end{code}

\begin{code}
\caption{Βελτίωση του υπολογισμού της \texttt{viz} με \texttt{numpy}}\label{lst:viz}
\begin{pythoncode}
def print_viz(skeleton, resolution, x, y):
    i, j = numpy.where(skeleton == 1)
    viz = zip(i * resolution + x, j * resolution + y)
    RvizHandler.printMarker(
        viz,
        1,  # Type: Arrow
        0,  # Action: Add
        "map",  # Frame
        "art_skeletonization",  # Namespace
        [0.5, 0, 0, 0.5],  # Color RGBA
        0.05  # Scale
    )
\end{pythoncode}
\end{code}
