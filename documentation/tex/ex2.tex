\section{Challenge 2: Απεικόνιση διαδρομής}\label{section:ex2}
Σε αυτό το κομμάτι καλούμαστε να μεταφέρουμε τις συντεταγμένες από το pose του ρομπότ στο σύστημα συντεταγμένων που χρησιμοποιεί το RViz.
Σαν pose αναφερόμαστε στον συνδυασμό της θέσης (position) και προσανατολισμού (orientation)~\cite{shapiro2001computer}.

Τα 2 συστήματα συντεταγμένων έχουν ίδιο orientation αλλά διαφορετική μονάδα μέτρησης (pixel και μέτρα) και αρχή των αξόνων.
Άρα, για τον μετασχηματισμό που πρέπει να πραγματοποιηθεί δεν χρειάζεται να κάνουμε rotation αλλά θέλουμε scaling και μετατόπιση (translation).

\sloppy Η \mintinline{python}!class RobotPerception! του αρχείου \texttt{robot\_perception.py} περιέχει τη πληροφορία που χρειαζόμαστε για τον μετασχηματισμό.
Το πεδίο \mintinline{python}!RobotPerception.resolution! κρατάει την αναλογία μέτρα / pixel του occupancy grid map (ο χάρτης του RViz)
ενώ το πεδίο \mintinline{python}!RobotPerception.origin! κρατάει το translation μεταξύ των αρχών των 2 συστημάτων συντεταγμένων \textbf{μετά} τον scaling μετασχηματισμό.

Ο μετασχηματισμός που πρέπει να πραγματοποιηθεί είναι:
\begin{align}
    \mathbf{c'}  & = \mathbf{S} \mathbf{c}      \\
    \mathbf{c''} & = \mathbf{c'} + \mathbf{c_0}
\end{align}
όπου $\mathbf{S} = \begin{bmatrix}
resolution & 0 \\
0 & resolution
\end{bmatrix}$ και $\mathbf{c_0} = origin$.
Σε κώδικα:
\begin{code}
\caption{Μεταφορά συστήματος συντεταγμένων}
\begin{pythoncode}
ps.pose.position.x = p[0] * self.robot_perception.resolution + self.robot_perception.origin['x']
ps.pose.position.y = p[1] * self.robot_perception.resolution + self.robot_perception.origin['y']
\end{pythoncode}
\end{code}
