\section{Challenge 5: Εξυπνότερη επιλογή υποστόχου}
Καλούμαστε να βελτιώσουμε την επιλογή υποστόχου.
Μέχρι τώρα, το ρομπότ προσπαθούσε να επισκεφθεί διαδοχικά όλους τους υποστόχους μιας διαδρομής.
Μια βελτίωση σε αυτή τη στρατηγική μπορεί να γίνει για τη περίπτωση που το ρομπότ βρίσκεται ήδη σε έναν επόμενο υποστόχο.
Εδώ, μπορούμε να παραλείψουμε τους προηγούμενους υποστόχους και να συνεχίσουμε τη διαδρομή από εκείνο το σημείο.

\sloppy Θεωρούμε ότι το ρομπότ βρίσκεται καλύπτει έναν υποστόχο αν η απόστασή του με αυτόν είναι 7 pixel.
Για τον νέο αλγόριθμο που ακολουθείται: για κάθε υποστόχο στη λίστα μετά τον τρέχων υποστόχο
(\mintinline{python}!self.subtargets[self.next_subtarget::]!)
υπολογίζουμε τη θέση του στην αρχική λίστα
\mintinline{python}!idx = len(self.subtargets) - 1 - idx!
και την απόστασή του από το ρομπότ
\mintinline{python}!dist = math.hypot(rx - st[0], ry - st[1])!.
Αν \mintinline{python}!dist < 7! θέτουμε ως επόμενο υποστόχο τον \mintinline{python}!idx + 1! και σταματάμε εκεί την επανάληψη.
Αν δεν υπάρξει καμία αλλαγή, το ρομπότ συνεχίζει τη πορεία του προς τον παλιό υποστόχο.

\begin{code}
\caption{Εξυπνότερη επιλογή υποστόχου}
\begin{pythoncode}
for idx, st in enumerate(self.subtargets[self.next_subtarget::][::-1]):
    # Right now, idx refers to the sliced & reversed array, fix it.
    idx = len(self.subtargets) - 1 - idx
    assert idx >= self.next_subtarget
    dist = math.hypot(rx - st[0], ry - st[1])
    if dist < 7:
        self.next_subtarget = idx + 1
        self.counter_to_next_sub = self.count_limit
        if self.next_subtarget == len(self.subtargets):
            self.target_exists = False
        break
\end{pythoncode}
\end{code}
